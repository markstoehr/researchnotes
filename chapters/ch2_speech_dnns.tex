\chapter{Speech Recognition with Deep Neural Networks}

\section{Combining time-and frequency-domain convolution in convolutional neural network-based phone recognition}

A paper by L{\'a}szl{\'o} T{\'o}th that develops a convolutional network for phone recognition
on the TIMIT database.


\section{Convolutional Deep Maxout Networks for Phone Recognition}
In this section we are writing up an explanation of 
\cite{toth2014convolutional}.  The paper use mel features
(as described in Chapter~\ref{sec:mel-spectral-features})
as the basis for the system.


\subsection{Convolutional Neural Networks}

The convolutional network uses as its base input a 17-frame by
7 mel-channel time-mel block: i.e. 7 rows by 17 columns in the
time frequency representation.  Pooling and convolution are central
to convolutional neural networks and they are usually performed
along the frequency axis.  An $r$-pooling network will move the $7\times 17$
along filter along $7$ adjacent mel feature blocks and takes the maximum over
them.
