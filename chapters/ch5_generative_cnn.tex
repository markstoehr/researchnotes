\chapter{Generative Convolutional Neural Networks}

In this chapter we develop the idea of a generative convolutional neural
network.  Doing so will allow us to use a natural generative pre-training
algorithm so that we do not need the same quantity of labeled data
that deep neural networks normally require.

\section{Fixed-length speech samples}

We begin our discussion with fixed-length vectors of speech signals
$\mathbf{X}\in\mathbb{R}^{N\times D}$ which denotes $N$ vectors
of dimension $D$.  We use Numpy-style slicing syntax so that
$\mathbf{X}(n_1:n_2,d_1:d_2)\in \mathbb{R}^{(n_2-n_1)\times(d_2-d_1)}$
refers to the matrix
\begin{equation}
\begin{bmatrix}
  \mathbf{X}(n_1,d_1) & \mathbf{X}(n_1,d_1+1) & \cdots & \mathbf{X}(n_1,d_2-1)\\
  \mathbf{X}(n_1+1,d_1) & \mathbf{X}(n_1+1,d_1+1) & \cdots & \mathbf{X}(n_1+1,d_2-1)\\
  \vdots & \vdots & \ddots & \vdots \\
  \mathbf{X}(n_2-1,d_1) & \mathbf{X}(n_2-1,d_1+1) & \cdots & \mathbf{X}(n_2-1,d_2-1)
\end{bmatrix}
\end{equation}
and we denote the index set for a vector of length $D$ as $[D]=\{0,1,\ldots,D-1\}$. We will consider a sequence of models for these data.  Initially we will begin with 
a mixture model and progressively we will develop towards the convolutional neural network model.

\subsection{Circulant Gaussian Models}

We may model these vectors using circulant Gaussians $\mathcal{N}_D(\mathbf{0},\mathbf{\Sigma})$ where
$\tilde{\boldsymbol{\Sigma}}\in\Circ_D$ where $\Circ_D$ is the set of circulant matrices over $\mathbb{R}^{D\times D}$.
Since $\boldsymbol{\Sigma}$ is circulant we may write it as $F_D^*\diag(\tilde{\boldsymbol{\sigma}})F_D$ for some 
real vector $\tilde{\boldsymbol{\sigma}}\in\mathbb{R}^D$. Since $\boldsymbol{\Sigma}$ is a real symmetric
 positive definite circulant matrix we may write it as $\boldsymbol{\Sigma}=\Circ(\boldsymbol{\sigma})$
where $\boldsymbol{\sigma}$ is the first column of $\boldsymbol{\Sigma}$.  Due to the symmetry and circularity of
$\boldsymbol{\Sigma}$ this means that $\boldsymbol{\sigma}$ is symmetric so that for $1 < d < D$ we have
$\boldsymbol{\sigma}(d)=\boldsymbol{\sigma}(D-d)$ and the same holds for $\tilde{\boldsymbol{\sigma}}=\mathbf{F}\boldsymbol{\sigma}$.

The circulant model is given by a single vector specifying the spectral distribution
